\documentclass[serif]{beamer}
\usepackage{graphicx} % Required for inserting images
\usepackage{babel}
\renewcommand{\figurename}{Fig.}
\usepackage[labelsep=endash]{caption}
\usepackage[dvipsnames]{color, xcolor}
\usepackage{subcaption}
\usepackage{tikz,pgfplots}
\usepackage{tikzpagenodes}
\usetikzlibrary{datavisualization.formats.functions,backgrounds,calc}
\usetikzlibrary{decorations.pathreplacing}
\usepackage{pgfplotstable}
\usetikzlibrary{pgfplots.colormaps}
\pgfplotsset{compat=1.9}
\usepgfplotslibrary{units,groupplots}
\usepgfplotslibrary{patchplots}
\usepgfplotslibrary{fillbetween}
\usetikzlibrary{patterns}
\usetikzlibrary{arrows.meta}

\usepackage[utf8]{inputenc}
\usepackage[TS1,T1]{fontenc}
\usepackage{fourier, heuristica}
\usepackage{array, booktabs}
\usepackage{graphicx}
\usepackage[x11names,table]{xcolor}
\usepackage{caption}

\newcommand{\foo}{\makebox[0pt]{\textbullet}\hskip-0.5pt\vrule width 1pt\hspace{\labelsep}}


\title{DAC Presentation; Winter 2025}
\author{Joseph McLaughlin}
\date{March 2025}

\setbeamertemplate{navigation symbols}{}
\setbeamercolor{normal text}{fg=black}
%\setbeamercolor{frametitle}{fg=black}

\begin{document}

\maketitle

\section{Introduction}

\begin{frame}{External work \& service}
    \begin{center}        
        \begin{itemize}
            \setlength{\itemindent}{4em}
            \item[May 2024] Artifact evaluator for SC '24.
            \item[June 2024] On-site internship at LANL.
            \item[November 2024] Invited Guest Lecturer, Cal Poly, CSC 469.
            \item[February 2025] Co-chaired UO-AMD Workshop (45 attendees).
            \item[March 2025] Lecturer, University of Oregon, CS 330.
            \item[GE-R] SNL LDRD on performance of tensor methods.  
        \end{itemize}
    \end{center}
\end{frame}

\begin{frame}{Current direction}
A comprehensive ``end-to-end'' understanding of performance for applications of PDE solvers. \\[1em]

PDE application hierarchy 
\begin{itemize}
    \item Meshing techniques, timesteppers. 
    \item Properties of the numerical method, \emph{e.g.}, stability, order-of-accuracy.
    \item Solver methods, preconditioners. 
    \item Linear algebra kernel implementations.
    \item Hardware.
\end{itemize}
\end{frame}

\begin{frame}{Current direction}
Components of these problems are often studied in isolation or \emph{at least} without an explicit consideration for other components of the problem. \\ [1em]
\begin{itemize}
    \item[\textbf{RQ}] How should we interoperate performance results that are studied in a narrow context?
    \item[\textbf{RQ}] Are there any major holes in our understanding of these systems?   
    \item[\textbf{RQ}] How can we demonstrate performance for a wider context of problems? 
\end{itemize}
\end{frame}

\begin{frame}{Current research: Manuscript A}
    \begin{center}
        \emph{``SMOOTH: Shared Memory Optimal Orthogonal Tiling for Hybridized PDEs''}
    \end{center}

    \textbf{Motivation} --- hybridization allows us to decompose this problem into smaller computational blocks. The ubiquitous strategy in the literature is to use very small blocks, why is this? 

    \textbf{Key contributions}
    \begin{itemize}
        \item A hardware-aware performance model. 
        \item Analytical and empirical results show that most problems are meshed too finely and that the ideal mesh is hardware dependent. 
        \item A performant implementation for Intel CPUs.
        \item Notable speedup over existing work that validate our model (up $92\times$ depending on the solver). 
    \end{itemize}
\end{frame}

\begin{frame}{Manuscript A: Updates}

    Submitted to PPoPP'25 (CORE Rank: A). \\
    \begin{center}
    Reviews: \emph{{\color{ForestGreen} accept}, {\color{YellowGreen}weak accept}, {\color{YellowOrange} weak reject}, {\color{YellowOrange} weak reject}. \\ Familiarity: moderate, moderate, low, low.}
    \end{center}
    Final decision: reject. \\

    Key feedback:
    \begin{itemize}
        \item Clarify contributions. 
        \item Additional performance results.
        \item Improve figures.
    \end{itemize}
\end{frame}

\begin{frame}[fragile]{Manuscript A: Updates}
    \begin{figure}[tp]
        \centering
            \begin{subfigure}[t]{\textwidth}
                \centering
                
\begin{tikzpicture}[scale=0.6]
\begin{axis}[
    title = {\footenotesize Perf. model on Vu et al. 2022.},
    colorbar,
    colorbar style={
        ylabel style = {
            %text width = {1.1in},
            %align = center
        },
        ytick = {-3, 0, 3},
        yticklabels = {$0.3\times$, $1\times$, $3\times$},
        ylabel={Speedup},
    },
    colormap/viridis,
    % view={135}{90},
    height = 1.2in,
    width = \columnwidth - 7em,
    view={180}{90},
    %xlabel={Sub-problem size $(n^2)$}, 
    ylabel = {Platform},
    ylabel style={rotate=90},
    axis x line* = bottom,
    axis y line* = left,
    xticklabel style = {
        yshift={-0.6em},
        /pgf/number format/fixed,
        anchor = north},
    yticklabel style = {
        %rotate=90,
        xshift={-1.4em},
        yshift={0.7em},
        /pgf/number format/fixed,
        anchor = north},
    scaled x ticks=false,
    scaled y ticks=false,
    x dir = reverse,
    y dir = reverse,
    y axis line style={draw opacity=0},
    x axis line style={draw opacity=0},
    %mesh/cols = 20,
    mesh/ordering=x varies,
    domain z = {0, 30},
    ytick style={draw=none},
    ytick = {1, 2, 3},
    xtick = {36, 100, 200, 300},
    yticklabels = {SR, IL, BW},
    declytickare function={f(\x,\y)=-((((8 * (\x*\y) / \y) * (16 * \y)))/2800)) - (2 * ((((15 * ((\x*\y)/\y)^(1.75)) * \y)/69000)));}
    ]
\addplot3[
    scatter,
    mark=square*,
    mesh/rows = 2,
    mark size=5pt,
    % patch refines = 1,
    shader = interp,
    draw opacity = 0] 
    coordinates {
( 10 , 1 , -2.1427699427520652 ) 
( 20 , 1 , -1.2268169345694968 ) 
( 30 , 1 , -0.5495090703443006 ) 
( 40 , 1 , -0.0 ) 
( 50 , 1 , 0.4651303310414967 ) 
( 60 , 1 , 0.8683923006973293 ) 
( 70 , 1 , 1.223301108236341 ) 
( 80 , 1 , 1.5387559174983023 ) 
( 90 , 1 , 1.8210030835162927 ) 
( 100 , 1 , 2.0746380215058036 ) 
( 110 , 1 , 2.303165882794689 ) 
( 120 , 1 , 2.5093378847684367 ) 
( 130 , 1 , 2.695364326638927 ) 
( 140 , 1 , 2.8630555374014044 ) 
( 150 , 1 , 3.0139185602367835 ) 
( 160 , 1 , 3.1492255090309618 ) 
( 170 , 1 , 3.2700631563834737 ) 
( 180 , 1 , 3.3773697129185987 ) 
( 190 , 1 , 3.4719626384609237 ) 
( 200 , 1 , 3.554560031985332 ) 
( 210 , 1 , 3.6257973323404302 ) 
( 220 , 1 , 3.6862405340281734 ) 
( 230 , 1 , 3.7363967721178115 ) 
( 240 , 1 , 3.7767228928425656 ) 
( 250 , 1 , 3.807632462127075 ) 
( 260 , 1 , 3.829501548616765 ) 
( 270 , 1 , 3.8426735350201024 ) 
( 280 , 1 , 3.847463151483921 ) 
( 290 , 1 , 3.844159880509168 ) 
( 300 , 1 , 3.8330308499722854 ) 
};    

\addplot3[
    line width = 0.75pt,
    draw = black,
    mark size = 2pt,
    mark = +]
    coordinates {( 280, 1 , 0 )};

\addplot3[
    scatter,
    mark=square*,
    mesh/rows = 2,
    mark size=5pt,
    % patch refines = 1,
    shader = interp,
    draw opacity = 0] 
    coordinates {
( 10 , 2 , -1.5585666425484312 ) 
( 20 , 2 , -0.6590185838312612 ) 
( 30 , 2 , -0.0 ) 
( 40 , 2 , 0.528674580602484 ) 
( 50 , 2 , 0.9702781621055534 ) 
( 60 , 2 , 1.3473051301853278 ) 
( 70 , 2 , 1.673298489384245 ) 
( 80 , 2 , 1.957200988572394 ) 
( 90 , 2 , 2.2053076655138115 ) 
( 100 , 2 , 2.4222632535747524 ) 
( 110 , 2 , 2.611620929281777 ) 
( 120 , 2 , 2.776177834999327 ) 
( 130 , 2 , 2.9181878094152625 ) 
( 140 , 2 , 3.0395022990068687 ) 
( 150 , 2 , 3.1416671235303006 ) 
( 160 , 2 , 3.2259909659429153 ) 
( 170 , 2 , 3.2935951130483865 ) 
( 180 , 2 , 3.3454503898962207 ) 
( 190 , 2 , 3.3824051201959193 ) 
( 200 , 2 , 3.405206655838966 ) 
( 210 , 2 , 3.4145182060947965 ) 
( 220 , 2 , 3.4109321705827034 ) 
( 230 , 2 , 3.394980830556449 ) 
( 240 , 2 , 3.3671450158067495 ) 
( 250 , 2 , 3.3278612002995756 ) 
( 260 , 2 , 3.277527364007371 ) 
( 270 , 2 , 3.2165078755895804 ) 
( 280 , 2 , 3.145137590428271 ) 
( 290 , 2 , 3.0637253142377086 ) 
( 300 , 2 , 2.972556749453892 ) 
    };    

\addplot3[
    line width = 0.75pt,
    draw = black,
    mark size = 2pt,
    mark = +]
    coordinates {( 210  , 2 , 0 )};

\addplot3[
    scatter,
    mark=square*,
    mesh/rows = 2,
    mark size=5pt,
    % patch refines = 1,
    shader = interp,
    draw opacity = 0] 
    coordinates {
    ( 10 , 3 , -1.4466682490412053 ) 
( 20 , 3 , -0.5954613776641748 ) 
( 30 , 3 , -0.0 ) 
( 40 , 3 , 0.448793770764766 ) 
( 50 , 3 , 0.7940982501714711 ) 
( 60 , 3 , 1.0585532817513732 ) 
( 70 , 3 , 1.2559153397566636 ) 
( 80 , 3 , 1.3953558500654273 ) 
( 90 , 3 , 1.4833951592315264 ) 
( 100 , 3 , 1.5248932487485822 ) 
( 110 , 3 , 1.5236063899973118 ) 
( 120 , 3 , 1.4825224532416748 ) 
( 130 , 3 , 1.4040741912732857 ) 
( 140 , 3 , 1.290280989784387 ) 
( 150 , 3 , 1.142846543562781 ) 
( 160 , 3 , 0.9632282343595477 ) 
( 170 , 3 , 0.7526876974377483 ) 
( 180 , 3 , 0.5123285065725902 ) 
( 190 , 3 , 0.24312480866641728 ) 
( 200 , 3 , -0.054056544582281596 ) 
( 210 , 3 , -0.37843863089888075 ) 
( 220 , 3 , -0.7293206464125346 ) 
( 230 , 3 , -1.1060667880548936 ) 
( 240 , 3 , -1.5080972058207465 ) 
( 250 , 3 , -1.9348805661865143 ) 
( 260 , 3 , -2.3859278843143317 ) 
( 270 , 3 , -2.8607873660995597 ) 
( 280 , 3 , -3.3590400618346727 ) 
( 290 , 3 , -3.8802961780476153 ) 
( 300 , 3 , -4.424191927518617 ) 
};

\addplot3[
    line width = 0.75pt,
    draw = black,
    mark size = 2pt,
    mark = +]
    coordinates {( 110 , 3 , 0.0 ) };

\addplot3[
    line width = 0.5pt,
    fill = black,
    mark size = 2pt,
    mark = square*]
    coordinates {( 36 , 1 , 0 ) ( 36 , 2 , 0 ) ( 36 , 3 , 0 )};
\end{axis}
\end{tikzpicture}
            \end{subfigure}
            \begin{subfigure}[t]{\textwidth}
                \centering
            \begin{tikzpicture}[scale=0.6]
\begin{axis}[
    title = {\footnotesize Perf. model on Rhebergen 2022.},
    colorbar,
    colorbar style={
        ylabel style = {
            %text width = {1.1in},
            %align = center
        },
        ytick = {0, 10, 20},
        yticklabels = {$1\times$, $10\times$, $20\times$},
        ylabel={Speedup},
    },
    colormap/viridis,
    % view={135}{90},
    height = 1.2in,
    width = \columnwidth - 7em,
    view={180}{90},
    xlabel={Sub-problem size $(n^2)$}, 
    ylabel = {Platform},
    ylabel style={rotate=90},
    axis x line* = bottom,
    axis y line* = left,
    xticklabel style = {
        yshift={-0.6em},
        /pgf/number format/fixed,
        anchor = north},
    yticklabel style = {
        %rotate=90,
        xshift={-1.4em},
        yshift={0.7em},
        /pgf/number format/fixed,
        anchor = north},
    legend columns = 2,
    legend style={
        at={(0.5,-1)},
        anchor = north,
        font=\small},
    legend entries={Author, Optimal
                      },
    scaled x ticks=false,
    scaled y ticks=false,
    x dir = reverse,
    y dir = reverse,
    y axis line style={draw opacity=0},
    x axis line style={draw opacity=0},
    %mesh/cols = 20,
    mesh/ordering=x varies,
    domain z = {0, 30},
    ytick style={draw=none},
    ytick = {1, 2, 3},
    xtick = {53, 10000, 30000, 50000},
    yticklabels = {SR, IL, BW},
    xticklabels = {53, 1e4, 3e4, 5e4},
    declytickare function={f(\x,\y)=-((((8 * (\x*\y) / \y) * (16 * \y)))/2800)) - (2 * ((((15 * ((\x*\y)/\y)^(1.75)) * \y)/69000)));}
    ]
\addplot3[
    scatter,
    mark=square*,
    mesh/rows = 2,
    mark size=5pt,
    % patch refines = 1,
    shader = interp,
    draw opacity = 0] 
    coordinates {
( 10 , 1, -1.5367533068014083 ) 
( 260 , 1, 2.148697187550127 ) 
( 510 , 1, 3.452017021583572 ) 
( 760 , 1, 4.393856418990303 ) 
( 1010 , 1, 5.16200441894199 ) 
( 1260 , 1, 5.823967925175676 ) 
( 1510 , 1, 6.412792097486307 ) 
( 1760 , 1, 6.947461687136666 ) 
( 2010 , 1, 7.440032052447808 ) 
( 2260 , 1, 7.898705423872652 ) 
( 2510 , 1, 8.32935229014604 ) 
( 2760 , 1, 8.736340470960343 ) 
( 3010 , 1, 9.123021069967315 ) 
( 3260 , 1, 9.492029995088267 ) 
( 3510 , 1, 9.845483806715482 ) 
( 3760 , 1, 10.185111810058988 ) 
( 4010 , 1, 10.512348002137003 ) 
( 4260 , 1, 10.828396814771455 ) 
( 4510 , 1, 11.134281220194751 ) 
( 4760 , 1, 11.430878646044128 ) 
( 5010 , 1, 11.71894826692603 ) 
( 5260 , 1, 11.999152070104333 ) 
( 5510 , 1, 12.272071344062716 ) 
( 5760 , 1, 12.538219747093969 ) 
( 6010 , 1, 12.798053783009554 ) 
( 6260 , 1, 13.051981284954326 ) 
( 6510 , 1, 13.300368350572864 ) 
( 6760 , 1, 13.54354505990316 ) 
( 7010 , 1, 13.781810226828092 ) 
( 7260 , 1, 14.01543537612466 ) 
( 7510 , 1, 14.244668094685496 ) 
( 7760 , 1, 14.469734872983796 ) 
( 8010 , 1, 14.690843528274897 ) 
( 8260 , 1, 14.90818528226081 ) 
( 8510 , 1, 15.121936551477509 ) 
( 8760 , 1, 15.332260497417847 ) 
( 9010 , 1, 15.539308374583703 ) 
( 9260 , 1, 15.743220707699022 ) 
( 9510 , 1, 15.944128323772098 ) 
( 9760 , 1, 16.142153260259803 ) 
( 10010 , 1, 16.337409567007597 ) 
( 10260 , 1, 16.530004016740538 ) 
( 10510 , 1, 16.720036736512675 ) 
( 10760 , 1, 16.9076017705855 ) 
( 11010 , 1, 17.092787583605638 ) 
( 11260 , 1, 17.275677511629926 ) 
( 11510 , 1, 17.456350167444775 ) 
( 11760 , 1, 17.634879805705992 ) 
( 12010 , 1, 17.811336652655108 ) 
( 12260 , 1, 17.985787204516022 ) 
( 12510 , 1, 18.15829449812749 ) 
( 12760 , 1, 18.328918356899514 ) 
( 13010 , 1, 18.497715614784703 ) 
( 13260 , 1, 18.66474032061558 ) 
( 13510 , 1, 18.83004392486831 ) 
( 13760 , 1, 18.993675450662128 ) 
( 14010 , 1, 19.155681650588562 ) 
( 14260 , 1, 19.316107150777384 ) 
( 14510 , 1, 19.47499458344446 ) 
( 14760 , 1, 19.63238470902502 ) 
( 15010 , 1, 19.788316528874915 ) 
( 15260 , 1, 19.942827389412663 ) 
( 15510 , 1, 20.095953078482864 ) 
( 15760 , 1, 20.247727914638844 ) 
( 16010 , 1, 20.398184829968333 ) 
( 16260 , 1, 20.54735544702393 ) 
( 16510 , 1, 20.695270150362052 ) 
( 16760 , 1, 20.841958153144446 ) 
( 17010 , 1, 20.98744755921209 ) 
( 17260 , 1, 21.131765421000992 ) 
( 17510 , 1, 21.274937793635694 ) 
( 17760 , 1, 21.416989785503034 ) 
( 18010 , 1, 21.557945605582418 ) 
( 18260 , 1, 21.69782860778224 ) 
( 18510 , 1, 21.836661332510985 ) 
( 18760 , 1, 21.974465545690244 ) 
( 19010 , 1, 22.111262275399433 ) 
( 19260 , 1, 22.24707184632515 ) 
( 19510 , 1, 22.381913912173676 ) 
( 19760 , 1, 22.51580748619175 ) 
( 20010 , 1, 22.648770969928478 ) 
( 20260 , 1, 22.780822180360904 ) 
( 20510 , 1, 22.91197837549501 ) 
( 20760 , 1, 23.042256278545544 ) 
( 21010 , 1, 23.171672100790275 ) 
( 21260 , 1, 23.30024156318555 ) 
( 21510 , 1, 23.42797991682461 ) 
( 21760 , 1, 23.55490196231351 ) 
( 22010 , 1, 23.68102206813316 ) 
( 22260 , 1, 23.80635418805219 ) 
( 22510 , 1, 23.93091187764924 ) 
( 22760 , 1, 24.054708310000137 ) 
( 23010 , 1, 24.177756290580565 ) 
( 23260 , 1, 24.300068271431893 ) 
( 23510 , 1, 24.4216563646338 ) 
( 23760 , 1, 24.542532355124823 ) 
( 24010 , 1, 24.66270771290937 ) 
( 24260 , 1, 24.766977909338102 ) 
( 24510 , 1, 24.806731617575913 ) 
( 24760 , 1, 24.845897457155438 ) 
( 25010 , 1, 24.88448448213858 ) 
( 25260 , 1, 24.92250151907713 ) 
( 25510 , 1, 24.959957174910883 ) 
( 25760 , 1, 24.99685984451778 ) 
( 26010 , 1, 25.033217717932942 ) 
( 26260 , 1, 25.06903878725666 ) 
( 26510 , 1, 25.1043308532648 ) 
( 26760 , 1, 25.13910153173957 ) 
( 27010 , 1, 25.17335825953308 ) 
( 27260 , 1, 25.207108300378508 ) 
( 27510 , 1, 25.24035875046122 ) 
( 27760 , 1, 25.273116543761585 ) 
( 28010 , 1, 25.305388457181287 ) 
( 28260 , 1, 25.3371811154635 ) 
( 28510 , 1, 25.36850099591674 ) 
( 28760 , 1, 25.399354432952315 ) 
( 29010 , 1, 25.42974762244414 ) 
( 29260 , 1, 25.459686625918952 ) 
( 29510 , 1, 25.48917737458558 ) 
( 29760 , 1, 25.518225673210246 ) 
( 30010 , 1, 25.546837203845094 ) 
( 30260 , 1, 25.575017529416865 ) 
( 30510 , 1, 25.60277209718211 ) 
( 30760 , 1, 25.630106242054417 ) 
( 31010 , 1, 25.657025189810255 ) 
( 31260 , 1, 25.68353406017796 ) 
( 31510 , 1, 25.70963786981566 ) 
( 31760 , 1, 25.73534153518242 ) 
( 32010 , 1, 25.760649875307994 ) 
( 32260 , 1, 25.785567614464465 ) 
( 32510 , 1, 25.810099384745207 ) 
( 32760 , 1, 25.834249728553992 ) 
( 33010 , 1, 25.858023101008484 ) 
( 33260 , 1, 25.881423872262197 ) 
( 33510 , 1, 25.90445632974718 ) 
( 33760 , 1, 25.927124680341507 ) 
( 34010 , 1, 25.949433052464485 ) 
( 34260 , 1, 25.971385498102382 ) 
( 34510 , 1, 25.992985994767444 ) 
( 34760 , 1, 26.014238447393097 ) 
( 35010 , 1, 26.03514669016744 ) 
( 35260 , 1, 26.055714488307945 ) 
( 35510 , 1, 26.07594553977934 ) 
( 35760 , 1, 26.095843476957036 ) 
( 36010 , 1, 26.1154118682378 ) 
( 36260 , 1, 26.134654219600527 ) 
( 36510 , 1, 26.15357397611806 ) 
( 36760 , 1, 26.17217452342262 ) 
( 37010 , 1, 26.19045918912599 ) 
( 37260 , 1, 26.208431244196717 ) 
( 37510 , 1, 26.2260939042956 ) 
( 37760 , 1, 26.24345033107087 ) 
( 38010 , 1, 26.260503633414785 ) 
( 38260 , 1, 26.27725686868316 ) 
( 38510 , 1, 26.293713043878597 ) 
( 38760 , 1, 26.30987511679925 ) 
( 39010 , 1, 26.325745997154378 ) 
( 39260 , 1, 26.34132854764719 ) 
( 39510 , 1, 26.356625585027025 ) 
( 39760 , 1, 26.371639881111285 ) 
( 40010 , 1, 26.386374163778335 ) 
( 40260 , 1, 26.400831117932608 ) 
( 40510 , 1, 26.415013386442652 ) 
( 40760 , 1, 26.42892357105284 ) 
( 41010 , 1, 26.442564233270136 ) 
( 41260 , 1, 26.4559378952262 ) 
( 41510 , 1, 26.469047040516244 ) 
( 41760 , 1, 26.481894115014825 ) 
( 42010 , 1, 26.49448152766984 ) 
( 42260 , 1, 26.50681165127484 ) 
( 42510 , 1, 26.51888682322123 ) 
( 42760 , 1, 26.53070934623012 ) 
( 43010 , 1, 26.542281489064987 ) 
( 43260 , 1, 26.553605487225738 ) 
( 43510 , 1, 26.564683543624554 ) 
( 43760 , 1, 26.57551782924432 ) 
( 44010 , 1, 26.586110483779898 ) 
( 44260 , 1, 26.596463616263264 ) 
( 44510 , 1, 26.606579305672316 ) 
( 44760 , 1, 26.61645960152481 ) 
( 45010 , 1, 26.62610652445665 ) 
( 45260 , 1, 26.635522066786166 ) 
( 45510 , 1, 26.644708193064258 ) 
( 45760 , 1, 26.653666840610743 ) 
( 46010 , 1, 26.662399920037647 ) 
( 46260 , 1, 26.670909315759417 ) 
( 46510 , 1, 26.679196886491074 ) 
( 46760 , 1, 26.687264465733993 ) 
( 47010 , 1, 26.695113862250007 ) 
( 47260 , 1, 26.702746860524137 ) 
( 47510 , 1, 26.71016522121634 ) 
( 47760 , 1, 26.717370681602603 ) 
( 48010 , 1, 26.724364956005296 ) 
( 48260 , 1, 26.73114973621375 ) 
( 48510 , 1, 26.737726691894846 ) 
( 48760 , 1, 26.74409747099413 ) 
( 49010 , 1, 26.750263700127363 ) 
( 49260 , 1, 26.756226984963646 ) 
( 49510 , 1, 26.761988910599175 ) 
( 49760 , 1, 26.767551041922676 ) 
( 50010 , 1, 26.772914923972735 ) 
( 50260 , 1, 26.77808208228687 ) 
( 50510 , 1, 26.783054023242723 ) 
( 50760 , 1, 26.78783223439187 ) 
( 51010 , 1, 26.79241818478597 ) 
( 51260 , 1, 26.79681332529573 ) 
( 51510 , 1, 26.801019088923006 ) 
( 51760 , 1, 26.805036891105914 ) 
( 52010 , 1, 26.808868130017416 ) 
( 52260 , 1, 26.81251418685715 ) 
( 52510 , 1, 26.815976426137464 ) 
( 52760 , 1, 26.81925619596266 ) 
( 53010 , 1, 26.82235482830273 ) 
( 53260 , 1, 26.825273639261415 ) 
( 53510 , 1, 26.828013929337757 ) 
( 53760 , 1, 26.830576983683073 ) 
( 54010 , 1, 26.832964072352006 ) 
( 54260 , 1, 26.835176450548513 ) 
( 54510 , 1, 26.837215358866604 ) 
( 54760 , 1, 26.839082023526228 ) 
( 55010 , 1, 26.840777656604285 ) 
( 55260 , 1, 26.842303456260662 ) 
( 55510 , 1, 26.84366060696017 ) 
( 55760 , 1, 26.8448502796893 ) 
( 56010 , 1, 26.845873632169145 ) 
( 56260 , 1, 26.846731809063805 ) 
( 56510 , 1, 26.847425942184586 ) 
( 56760 , 1, 26.847957150690014 ) 
( 57010 , 1, 26.84832654128233 ) 
( 57260 , 1, 26.848535208399515 ) 
( 57510 , 1, 26.84858423440386 ) 
( 57760 , 1, 26.84847468976683 ) 
( 58010 , 1, 26.84820763325022 ) 
( 58260 , 1, 26.847784112083705 ) 
( 58510 , 1, 26.84720516213923 ) 
( 58760 , 1, 26.8464718081018 ) 
( 59010 , 1, 26.845585063636968 ) 
( 59260 , 1, 26.844545931555363 ) 
( 59510 , 1, 26.84335540397371 ) 
( 59760 , 1, 26.842014462473294 ) 
};    

\addplot3[
    line width = 0.75pt,
    draw = black,
    mark size = 2pt,
    mark = +]
    coordinates {( 57510, 1 , 0 )};

\addplot3[
    scatter,
    mark=square*,
    mesh/rows = 2,
    mark size=5pt,
    % patch refines = 1,
    shader = interp,
    draw opacity = 0] 
    coordinates {
( 10 , 2, -1.4014007960182475 ) 
( 260 , 2, 2.020643021406965 ) 
( 510 , 2, 3.261458266497133 ) 
( 760 , 2, 4.161410148249622 ) 
( 1010 , 2, 4.896663786465783 ) 
( 1260 , 2, 5.530927828568455 ) 
( 1510 , 2, 6.095495060452972 ) 
( 1760 , 2, 6.60838515878779 ) 
( 2010 , 2, 7.081060140783002 ) 
( 2260 , 2, 7.5213287979162295 ) 
( 2510 , 2, 7.934785909066335 ) 
( 2760 , 2, 8.325597753751811 ) 
( 3010 , 2, 8.69696312142927 ) 
( 3260 , 2, 9.051399661843895 ) 
( 3510 , 2, 9.390930050336113 ) 
( 3760 , 2, 9.717207649190865 ) 
( 4010 , 2, 10.031604039036 ) 
( 4260 , 2, 10.335271643204603 ) 
( 4510 , 2, 10.62918957655208 ) 
( 4760 , 2, 10.91419789246062 ) 
( 5010 , 2, 11.191023618521779 ) 
( 5260 , 2, 11.46030086098877 ) 
( 5510 , 2, 11.722586546785863 ) 
( 5760 , 2, 11.978372904627378 ) 
( 6010 , 2, 12.228097472941315 ) 
( 6260 , 2, 12.472151207185812 ) 
( 6510 , 2, 12.7108851090175 ) 
( 6760 , 2, 12.944615693256512 ) 
( 7010 , 2, 13.173629531874798 ) 
( 7260 , 2, 13.398187058218891 ) 
( 7510 , 2, 13.618525773251934 ) 
( 7760 , 2, 13.834862964609009 ) 
( 8010 , 2, 14.047398025823933 ) 
( 8260 , 2, 14.256314445179942 ) 
( 8510 , 2, 14.461781519837059 ) 
( 8760 , 2, 14.663955840151944 ) 
( 9010 , 2, 14.862982580688767 ) 
( 9260 , 2, 15.058996627772526 ) 
( 9510 , 2, 15.252123568141275 ) 
( 9760 , 2, 15.44248055901767 ) 
( 10010 , 2, 15.630177096501232 ) 
( 10260 , 2, 15.815315696412148 ) 
( 10510 , 2, 15.99799249945615 ) 
( 10760 , 2, 16.178297810726196 ) 
( 11010 , 2, 16.356316582030228 ) 
( 11260 , 2, 16.53212884426649 ) 
( 11510 , 2, 16.705810096017505 ) 
( 11760 , 2, 16.87743165365241 ) 
( 12010 , 2, 17.047060967489827 ) 
( 12260 , 2, 17.214761907950994 ) 
( 12510 , 2, 17.38059502510736 ) 
( 12760 , 2, 17.544617784579785 ) 
( 13010 , 2, 17.706884782366416 ) 
( 13260 , 2, 17.867447940851246 ) 
( 13510 , 2, 18.02635668796673 ) 
( 13760 , 2, 18.18365812124392 ) 
( 14010 , 2, 18.33939715827746 ) 
( 14260 , 2, 18.49361667495275 ) 
( 14510 , 2, 18.646357632629297 ) 
( 14760 , 2, 18.797659195337005 ) 
( 15010 , 2, 18.947558837927538 ) 
( 15260 , 2, 19.096092446016932 ) 
( 15510 , 2, 19.243294408467772 ) 
( 15760 , 2, 19.38919770307963 ) 
( 16010 , 2, 19.53383397608639 ) 
( 16260 , 2, 19.677233615997935 ) 
( 16510 , 2, 19.819425822270276 ) 
( 16760 , 2, 19.960438669238624 ) 
( 17010 , 2, 20.10029916570635 ) 
( 17260 , 2, 20.239033310545242 ) 
( 17510 , 2, 20.376666144627478 ) 
( 17760 , 2, 20.5132217993808 ) 
( 18010 , 2, 20.64872354223082 ) 
( 18260 , 2, 20.78319381917102 ) 
( 18510 , 2, 20.916654294678285 ) 
( 18760 , 2, 21.049125889173848 ) 
( 19010 , 2, 21.180628814211257 ) 
( 19260 , 2, 21.311182605557022 ) 
( 19510 , 2, 21.440806154316583 ) 
( 19760 , 2, 21.569517736244403 ) 
( 20010 , 2, 21.697335039365598 ) 
( 20260 , 2, 21.824275190027123 ) 
( 20510 , 2, 21.95035477748506 ) 
( 20760 , 2, 22.075589877128124 ) 
( 21010 , 2, 22.199996072427997 ) 
( 21260 , 2, 22.323588475700973 ) 
( 21510 , 2, 22.44638174775846 ) 
( 21760 , 2, 22.56839011651788 ) 
( 22010 , 2, 22.6896273946408 ) 
( 22260 , 2, 22.810106996258984 ) 
( 22510 , 2, 22.929841952845976 ) 
( 22760 , 2, 23.048844928286407 ) 
( 23010 , 2, 23.16712823319209 ) 
( 23260 , 2, 23.28470383851049 ) 
( 23510 , 2, 23.401583388467742 ) 
( 23760 , 2, 23.517778212885343 ) 
( 24010 , 2, 23.633299338907456 ) 
( 24260 , 2, 23.748157502172297 ) 
( 24510 , 2, 23.8623631574606 ) 
( 24760 , 2, 23.975926488849034 ) 
( 25010 , 2, 24.08885741939806 ) 
( 25260 , 2, 24.201165620398758 ) 
( 25510 , 2, 24.312860520203873 ) 
( 25760 , 2, 24.423951312664883 ) 
( 26010 , 2, 24.534446965196555 ) 
( 26260 , 2, 24.644356226489574 ) 
( 26510 , 2, 24.753687633888454 ) 
( 26760 , 2, 24.862449520453712 ) 
( 27010 , 2, 24.970650021723472 ) 
( 27260 , 2, 25.07829708219085 ) 
( 27510 , 2, 25.185398461511177 ) 
( 27760 , 2, 25.291961740452408 ) 
( 28010 , 2, 25.39799432660215 ) 
( 28260 , 2, 25.503503459843316 ) 
( 28510 , 2, 25.608496217608767 ) 
( 28760 , 2, 25.712979519927483 ) 
( 29010 , 2, 25.816960134270353 ) 
( 29260 , 2, 25.920444680206714 ) 
( 29510 , 2, 26.023439633879963 ) 
( 29760 , 2, 26.12595133231046 ) 
( 30010 , 2, 26.22798597753455 ) 
( 30260 , 2, 26.329549640586603 ) 
( 30510 , 2, 26.430648265331413 ) 
( 30760 , 2, 26.531287672154225 ) 
( 31010 , 2, 26.631473561514053 ) 
( 31260 , 2, 26.731211517367083 ) 
( 31510 , 2, 26.830507010465432 ) 
( 31760 , 2, 26.929365401537165 ) 
( 32010 , 2, 27.027791944352344 ) 
( 32260 , 2, 27.12579178868033 ) 
( 32510 , 2, 27.223369983143098 ) 
( 32760 , 2, 27.235933233999777 ) 
( 33010 , 2, 27.234545475014695 ) 
( 33260 , 2, 27.232794912384463 ) 
( 33510 , 2, 27.230685705035373 ) 
( 33760 , 2, 27.228221933885745 ) 
( 34010 , 2, 27.225407603873 ) 
( 34260 , 2, 27.222246645913064 ) 
( 34510 , 2, 27.218742918795957 ) 
( 34760 , 2, 27.214900211019245 ) 
( 35010 , 2, 27.210722242561936 ) 
( 35260 , 2, 27.206212666601658 ) 
( 35510 , 2, 27.20137507117657 ) 
( 35760 , 2, 27.19621298079509 ) 
( 36010 , 2, 27.19072985799429 ) 
( 36260 , 2, 27.18492910485002 ) 
( 36510 , 2, 27.17881406444005 ) 
( 36760 , 2, 27.172388022261764 ) 
( 37010 , 2, 27.16565420760675 ) 
( 37260 , 2, 27.15861579489345 ) 
( 37510 , 2, 27.151275904959427 ) 
( 37760 , 2, 27.143637606314908 ) 
( 38010 , 2, 27.135703916358867 ) 
( 38260 , 2, 27.1274778025592 ) 
( 38510 , 2, 27.11896218359795 ) 
( 38760 , 2, 27.110159930483047 ) 
( 39010 , 2, 27.10107386762808 ) 
( 39260 , 2, 27.091706773900384 ) 
( 39510 , 2, 27.08206138363929 ) 
( 39760 , 2, 27.072140387645426 ) 
( 40010 , 2, 27.061946434141728 ) 
( 40260 , 2, 27.051482129707672 ) 
( 40510 , 2, 27.040750040187245 ) 
( 40760 , 2, 27.029752691571396 ) 
( 41010 , 2, 27.018492570856644 ) 
( 41260 , 2, 27.006972126879546 ) 
( 41510 , 2, 26.99519377112869 ) 
( 41760 , 2, 26.98315987853457 ) 
( 42010 , 2, 26.970872788238133 ) 
( 42260 , 2, 26.958334804338648 ) 
( 42510 , 2, 26.94554819662187 ) 
( 42760 , 2, 26.932515201268405 ) 
( 43010 , 2, 26.91923802154408 ) 
( 43260 , 2, 26.905718828471638 ) 
( 43510 , 2, 26.89195976148524 ) 
( 43760 , 2, 26.877962929067902 ) 
( 44010 , 2, 26.86373040937251 ) 
( 44260 , 2, 26.849264250827165 ) 
( 44510 , 2, 26.834566472724575 ) 
( 44760 , 2, 26.819639065797162 ) 
( 45010 , 2, 26.804483992777264 ) 
( 45260 , 2, 26.789103188943322 ) 
( 45510 , 2, 26.77349856265281 ) 
( 45760 , 2, 26.75767199586148 ) 
( 46010 , 2, 26.74162534463025 ) 
( 46260 , 2, 26.72536043961933 ) 
( 46510 , 2, 26.70887908657049 ) 
( 46760 , 2, 26.692183066777808 ) 
( 47010 , 2, 26.675274137546513 ) 
( 47260 , 2, 26.65815403264149 ) 
( 47510 , 2, 26.640824462724556 ) 
( 47760 , 2, 26.62328711578177 ) 
( 48010 , 2, 26.605543657540377 ) 
( 48260 , 2, 26.5875957318759 ) 
( 48510 , 2, 26.569444961210188 ) 
( 48760 , 2, 26.55109294689968 ) 
( 49010 , 2, 26.532541269614864 ) 
( 49260 , 2, 26.51379148971118 ) 
( 49510 , 2, 26.494845147591327 ) 
( 49760 , 2, 26.475703764059123 ) 
( 50010 , 2, 26.45636884066569 ) 
( 50260 , 2, 26.43684186004758 ) 
( 50510 , 2, 26.417124286257334 ) 
( 50760 , 2, 26.397217565086905 ) 
( 51010 , 2, 26.377123124383584 ) 
( 51260 , 2, 26.35684237435909 ) 
( 51510 , 2, 26.33637670789184 ) 
( 51760 , 2, 26.315727500822582 ) 
( 52010 , 2, 26.294896112243762 ) 
( 52260 , 2, 26.27388388478227 ) 
( 52510 , 2, 26.2526921448764 ) 
( 52760 , 2, 26.231322203046695 ) 
( 53010 , 2, 26.20977535416107 ) 
( 53260 , 2, 26.188052877694325 ) 
( 53510 , 2, 26.16615603798196 ) 
( 53760 , 2, 26.144086084468967 ) 
( 54010 , 2, 26.121844251953075 ) 
( 54260 , 2, 26.09943176082328 ) 
( 54510 , 2, 26.076849817293013 ) 
( 54760 , 2, 26.05409961362868 ) 
( 55010 , 2, 26.031182328373692 ) 
( 55260 , 2, 26.00809912656741 ) 
( 55510 , 2, 25.984851159960186 ) 
( 55760 , 2, 25.961439567223632 ) 
( 56010 , 2, 25.93786547415675 ) 
( 56260 , 2, 25.914129993887958 ) 
( 56510 , 2, 25.890234227073062 ) 
( 56760 , 2, 25.866179262089183 ) 
( 57010 , 2, 25.84196617522496 ) 
( 57260 , 2, 25.81759603086693 ) 
( 57510 , 2, 25.79306988168215 ) 
( 57760 , 2, 25.76838876879737 ) 
( 58010 , 2, 25.743553721974678 ) 
( 58260 , 2, 25.71856575978373 ) 
( 58510 , 2, 25.693425889770506 ) 
( 58760 , 2, 25.668135108623005 ) 
( 59010 , 2, 25.642694402333667 ) 
( 59260 , 2, 25.617104746358876 ) 
( 59510 , 2, 25.59136710577488 ) 
( 59760 , 2, 25.56548243543144 ) 
    };    

\addplot3[
    line width = 0.75pt,
    draw = black,
    mark size = 2pt,
    mark = +]
    coordinates {( 32760  , 2 , 0 )};

\addplot3[
    scatter,
    mark=square*,
    mesh/rows = 2,
    mark size=5pt,
    % patch refines = 1,
    shader = interp,
    draw opacity = 0] 
    coordinates {
( 10 , 3, -0.749216079029356 ) 
( 260 , 3, 1.4119850278013537 ) 
( 510 , 3, 2.393282161753186 ) 
( 760 , 3, 3.137409659152169 ) 
( 1010 , 3, 3.760046410887565 ) 
( 1260 , 3, 4.305602707806933 ) 
( 1510 , 3, 4.796697808268391 ) 
( 1760 , 3, 5.246695996942364 ) 
( 2010 , 3, 5.66426679535148 ) 
( 2260 , 3, 6.055410214929104 ) 
( 2510 , 3, 6.424481373078251 ) 
( 2760 , 3, 6.774759293864841 ) 
( 3010 , 3, 7.1087855294516675 ) 
( 3260 , 3, 7.428577084009593 ) 
( 3510 , 3, 7.735766330181406 ) 
( 3760 , 3, 8.031696349056473 ) 
( 4010 , 3, 8.31748790338438 ) 
( 4260 , 3, 8.594087719092359 ) 
( 4510 , 3, 8.862304077956097 ) 
( 4760 , 3, 9.122833571681777 ) 
( 5010 , 3, 9.37628155919499 ) 
( 5260 , 3, 9.623178048054637 ) 
( 5510 , 3, 9.863990191439665 ) 
( 5760 , 3, 10.099132242144162 ) 
( 6010 , 3, 10.328973568511191 ) 
( 6260 , 3, 10.553845174246824 ) 
( 6510 , 3, 10.77404504971651 ) 
( 6760 , 3, 10.989842600811663 ) 
( 7010 , 3, 11.201482342493563 ) 
( 7260 , 3, 11.4091870008751 ) 
( 7510 , 3, 11.613160135585066 ) 
( 7760 , 3, 11.813588370044538 ) 
( 8010 , 3, 12.01064329897934 ) 
( 8260 , 3, 12.204483128460565 ) 
( 8510 , 3, 12.395254092912623 ) 
( 8760 , 3, 12.583091685059445 ) 
( 9010 , 3, 12.768121728120985 ) 
( 9260 , 3, 12.95046131429436 ) 
( 9510 , 3, 13.130219629345575 ) 
( 9760 , 3, 13.307498679754449 ) 
( 10010 , 3, 13.482393936122673 ) 
( 10260 , 3, 13.654994904333174 ) 
( 10510 , 3, 13.825385634131539 ) 
( 10760 , 3, 13.993645173307124 ) 
( 11010 , 3, 14.15984797441896 ) 
( 11260 , 3, 14.324064259986203 ) 
( 11510 , 3, 14.48636035121147 ) 
( 11760 , 3, 14.646798964589204 ) 
( 12010 , 3, 14.805439480151096 ) 
( 12260 , 3, 14.962338184593538 ) 
( 12510 , 3, 15.117548492102207 ) 
( 12760 , 3, 15.27112114532303 ) 
( 13010 , 3, 15.423104398618804 ) 
( 13260 , 3, 15.573544185481381 ) 
( 13510 , 3, 15.722484271742344 ) 
( 13760 , 3, 15.869966396026758 ) 
( 14010 , 3, 16.016030398724126 ) 
( 14260 , 3, 16.160714340603267 ) 
( 14510 , 3, 16.304054612068896 ) 
( 14760 , 3, 16.44608603394706 ) 
( 15010 , 3, 16.586841950587996 ) 
( 15260 , 3, 16.7263543159901 ) 
( 15510 , 3, 16.864653773573647 ) 
( 15760 , 3, 17.00176973016714 ) 
( 16010 , 3, 17.137730424711542 ) 
( 16260 , 3, 17.272562992134823 ) 
( 16510 , 3, 17.406293522807115 ) 
( 16760 , 3, 17.538947117942417 ) 
( 17010 , 3, 17.67054794128089 ) 
( 17260 , 3, 17.80111926735175 ) 
( 17510 , 3, 17.93068352658933 ) 
( 17760 , 3, 18.05926234754931 ) 
( 18010 , 3, 18.09213529006928 ) 
( 18260 , 3, 18.09282770619063 ) 
( 18510 , 3, 18.092629224983494 ) 
( 18760 , 3, 18.09155794758921 ) 
( 19010 , 3, 18.089631372558593 ) 
( 19260 , 3, 18.08686642353952 ) 
( 19510 , 3, 18.083279475350622 ) 
( 19760 , 3, 18.07888637855457 ) 
( 20010 , 3, 18.07370248263536 ) 
( 20260 , 3, 18.06774265787512 ) 
( 20510 , 3, 18.06102131601879 ) 
( 20760 , 3, 18.05355242980702 ) 
( 21010 , 3, 18.045349551453448 ) 
( 21260 , 3, 18.036425830133364 ) 
( 21510 , 3, 18.026794028549652 ) 
( 21760 , 3, 18.016466538633225 ) 
( 22010 , 3, 18.005455396433526 ) 
( 22260 , 3, 17.993772296248945 ) 
( 22510 , 3, 17.98142860404459 ) 
( 22760 , 3, 17.96843537020043 ) 
( 23010 , 3, 17.95480334163031 ) 
( 23260 , 3, 17.940542973309604 ) 
( 23510 , 3, 17.92566443924575 ) 
( 23760 , 3, 17.910177642925106 ) 
( 24010 , 3, 17.894092227265634 ) 
( 24260 , 3, 17.877417584103952 ) 
( 24510 , 3, 17.860162863243126 ) 
( 24760 , 3, 17.84233698108573 ) 
( 25010 , 3, 17.82394862887513 ) 
( 25260 , 3, 17.805006280566694 ) 
( 25510 , 3, 17.78551820034884 ) 
( 25760 , 3, 17.765492449832678 ) 
( 26010 , 3, 17.744936894928244 ) 
( 26260 , 3, 17.72385921242379 ) 
( 26510 , 3, 17.702266896283028 ) 
( 26760 , 3, 17.68016726367625 ) 
( 27010 , 3, 17.657567460757395 ) 
( 27260 , 3, 17.634474468201255 ) 
( 27510 , 3, 17.61089510651228 ) 
( 27760 , 3, 17.586836041116506 ) 
( 28010 , 3, 17.56230378724728 ) 
( 28260 , 3, 17.537304714635216 ) 
( 28510 , 3, 17.511845052011154 ) 
( 28760 , 3, 17.48593089143218 ) 
( 29010 , 3, 17.45956819243799 ) 
( 29260 , 3, 17.432762786046865 ) 
( 29510 , 3, 17.40552037859755 ) 
( 29760 , 3, 17.37784655544521 ) 
( 30010 , 3, 17.34974678451755 ) 
( 30260 , 3, 17.321226419737812 ) 
( 30510 , 3, 17.292290704320813 ) 
( 30760 , 3, 17.262944773947055 ) 
( 31010 , 3, 17.233193659821367 ) 
( 31260 , 3, 17.20304229162044 ) 
( 31510 , 3, 17.172495500334264 ) 
( 31760 , 3, 17.141558021006233 ) 
( 32010 , 3, 17.11023449537663 ) 
( 32260 , 3, 17.07852947443247 ) 
( 32510 , 3, 17.046447420869406 ) 
( 32760 , 3, 17.013992711467818 ) 
( 33010 , 3, 16.981169639387563 ) 
( 33260 , 3, 16.9479824163848 ) 
( 33510 , 3, 16.914435174953436 ) 
( 33760 , 3, 16.880531970395054 ) 
( 34010 , 3, 16.846276782819803 ) 
( 34260 , 3, 16.811673519081083 ) 
( 34510 , 3, 16.776726014646904 ) 
( 34760 , 3, 16.741438035409907 ) 
( 35010 , 3, 16.705813279439386 ) 
( 35260 , 3, 16.66985537867636 ) 
( 35510 , 3, 16.633567900575244 ) 
( 35760 , 3, 16.596954349693128 ) 
( 36010 , 3, 16.560018169229096 ) 
( 36260 , 3, 16.522762742515695 ) 
( 36510 , 3, 16.485191394464028 ) 
( 36760 , 3, 16.447307392964095 ) 
( 37010 , 3, 16.409113950242762 ) 
( 37260 , 3, 16.370614224180137 ) 
( 37510 , 3, 16.331811319586233 ) 
( 37760 , 3, 16.29270828943954 ) 
( 38010 , 3, 16.253308136088364 ) 
( 38260 , 3, 16.213613812417172 ) 
( 38510 , 3, 16.173628222977943 ) 
( 38760 , 3, 16.133354225089057 ) 
( 39010 , 3, 16.092794629902098 ) 
( 39260 , 3, 16.05195220343787 ) 
( 39510 , 3, 16.010829667592553 ) 
( 39760 , 3, 15.96942970111568 ) 
( 40010 , 3, 15.927754940560025 ) 
( 40260 , 3, 15.885807981205003 ) 
( 40510 , 3, 15.843591377954038 ) 
( 40760 , 3, 15.801107646207502 ) 
( 41010 , 3, 15.758359262710897 ) 
( 41260 , 3, 15.715348666380315 ) 
( 41510 , 3, 15.672078259105172 ) 
( 41760 , 3, 15.628550406529063 ) 
( 42010 , 3, 15.584767438809944 ) 
( 42260 , 3, 15.540731651359565 ) 
( 42510 , 3, 15.496445305563533 ) 
( 42760 , 3, 15.451910629482036 ) 
( 43010 , 3, 15.40712981853234 ) 
( 43260 , 3, 15.362105036153407 ) 
( 43510 , 3, 15.31683841445315 ) 
( 43760 , 3, 15.27133205483904 ) 
( 44010 , 3, 15.22558802863232 ) 
( 44260 , 3, 15.179608377666831 ) 
( 44510 , 3, 15.133395114872307 ) 
( 44760 , 3, 15.086950224843143 ) 
( 45010 , 3, 15.040275664392656 ) 
( 45260 , 3, 14.993373363093678 ) 
( 45510 , 3, 14.946245223805743 ) 
( 45760 , 3, 14.89889312318904 ) 
( 46010 , 3, 14.851318912205748 ) 
( 46260 , 3, 14.803524416609328 ) 
( 46510 , 3, 14.755511437421799 ) 
( 46760 , 3, 14.707281751399261 ) 
( 47010 , 3, 14.65883711148668 ) 
( 47260 , 3, 14.610179247261268 ) 
( 47510 , 3, 14.561309865365569 ) 
( 47760 , 3, 14.512230649930503 ) 
( 48010 , 3, 14.462943262987894 ) 
( 48260 , 3, 14.413449344873875 ) 
( 48510 , 3, 14.363750514622488 ) 
( 48760 , 3, 14.313848370350458 ) 
( 49010 , 3, 14.263744489632858 ) 
( 49260 , 3, 14.21344042987025 ) 
( 49510 , 3, 14.162937728647453 ) 
( 49760 , 3, 14.112237904084136 ) 
( 50010 , 3, 14.061342455177428 ) 
( 50260 , 3, 14.01025286213693 ) 
( 50510 , 3, 13.95897058671205 ) 
( 50760 , 3, 13.907497072512378 ) 
( 51010 , 3, 13.85583374532049 ) 
( 51260 , 3, 13.803982013398208 ) 
( 51510 , 3, 13.751943267785993 ) 
( 51760 , 3, 13.699718882595958 ) 
( 52010 , 3, 13.647310215298237 ) 
( 52260 , 3, 13.594718607001278 ) 
( 52510 , 3, 13.541945382726261 ) 
( 52760 , 3, 13.488991851675426 ) 
( 53010 , 3, 13.435859307494692 ) 
( 53260 , 3, 13.382549028530889 ) 
( 53510 , 3, 13.329062278083281 ) 
( 53760 , 3, 13.275400304649931 ) 
( 54010 , 3, 13.221564342169192 ) 
( 54260 , 3, 13.167555610255445 ) 
( 54510 , 3, 13.11337531443082 ) 
( 54760 , 3, 13.059024646351336 ) 
( 55010 , 3, 13.00450478402908 ) 
( 55260 , 3, 12.949816892049238 ) 
( 55510 , 3, 12.894962121783205 ) 
( 55760 , 3, 12.839941611597029 ) 
( 56010 , 3, 12.784756487055883 ) 
( 56260 , 3, 12.729407861124177 ) 
( 56510 , 3, 12.673896834361866 ) 
( 56760 , 3, 12.618224495116804 ) 
( 57010 , 3, 12.562391919713379 ) 
( 57260 , 3, 12.506400172637111 ) 
( 57510 , 3, 12.450250306715873 ) 
( 57760 , 3, 12.393943363297666 ) 
( 58010 , 3, 12.337480372424547 ) 
( 58260 , 3, 12.28086235300363 ) 
( 58510 , 3, 12.224090312974468 ) 
( 58760 , 3, 12.167165249473234 ) 
( 59010 , 3, 12.11008814899397 ) 
( 59260 , 3, 12.052859987546594 ) 
( 59510 , 3, 11.995481730811864 ) 
( 59760 , 3, 11.93795433429359 )  
};

\addplot3[
    line width = 0.75pt,
    draw = black,
    mark size = 2pt,
    mark = +]
    coordinates {( 18260 , 3 , 0.0 ) };

\addplot3[
    line width = 0.5pt,
    fill = black,
    mark size = 2pt,
    mark = square*]
    coordinates {( 36 , 1 , 0 ) ( 36 , 2 , 0 ) ( 36 , 3 , 0 )};

\addlegendimage{mark=square*, only marks};
\addlegendimage{mark=o*, only marks};

\end{axis}
\end{tikzpicture}
        \end{subfigure}
        \caption{Performance model applied to prior arts, predicting potential speed up for various CPUs.}
    \end{figure}
\end{frame}

\begin{frame}[fragile]{Manuscript A: Submission cycle 1}
    \begin{figure}[tp]
        \centering
        \begin{tikzpicture}[scale=0.7, trim axis left,trim axis right]
    \begin{axis}[
        ybar,
        every legend image post/.append style={
            scale=2
        },
        legend style = {
            at={(1.1, -0.35)},
            anchor = north east,
            legend columns = 2},
        height=2in,
        width=\columnwidth - 2em,
        axis x line*=none,
        axis y line*=right,
        xtick = {1, 2, 3, 4},
        xticklabels = {$1e6$, $2e6$, $4e6$, $8e6$},
        domain = 1:64,
        range = 0:100,
        xlabel={\footnotesize Global volume size ($\bar{n}^2$)},
        log ticks with fixed point,
        %ymode = log,
        %log origin y = infty,
        %grid=both,
        %grid style={line width=.1pt, draw=gray!10},
        %major grid style={line width=.2pt,draw=gray!50},
        xshift=-10em,
        ylabel={Execution time (s)}]
        \addplot[ybar,
            %every node near coord/.append style={
            %    font=\footnotesize, 
            %    yshift = 2em,
            %    xshift = -0.5em,
            %}, 
            fill=red!30,
            bar width=0.5em,
            postaction={
                pattern=north east lines,
                pattern color=black!60
            }] 
            coordinates {
                (1, 0.049) 
                (2, 0.12) 
                (3, 0.308) 
                (4, 1.32)};
        \addplot[ybar,
            %nodes near coords,
            %every node near coord/.append style={
            %    font=\footnotesize, 
            %    yshift = 2em,
            %    xshift = 0.5em,
            %}, 
            fill=violet!30,
            bar width=0.5em,
            postaction={
                pattern=north west lines,
                pattern color=black!60
            }] 
            coordinates {
                (1, 0.053) 
                (2, 0.141) 
                (3, 0.364) 
                (4, 1.51)};
            
        \addplot[ybar,    
            fill=green!30,
            bar width=0.5em] 
            coordinates {
                (1, 0.58) 
                (2, 0.9) 
                (3, 1.32)};

                \addplot[ybar,
            bar width=0.5em,
            fill=orange!30,
            postaction={
                pattern=grid,
                pattern color=black!60
            }] 
            coordinates {
                (1, 0.5) 
                (2, 1.12) 
                (3, 2.1)};
                

                
        \addlegendentry{\footnotesize SR Hyb.}
        \addlegendentry{\footnotesize IL Hyb.}
        \addlegendentry{\footnotesize SR CG/QR}
        \addlegendentry{\footnotesize IL CG/QR}
    \end{axis}
    \begin{axis}[
        legend style = {
            at={(-0.1, -0.35)},
            anchor = north west,
            legend columns = 1},
        height=2in,
        width=\columnwidth - 2em,
        axis x line*=bottom,
        axis y line*=left,
        xtick = {1, 2, 3, 4},
        axis x line=none,
        domain = 1:64,
        range = 0:100,
        xlabel={\footnotesize Problem size ($\bar{n^2}$)},
        log ticks with fixed point,
        %grid=both,
        %grid style={line width=.1pt, draw=gray!10},
        %major grid style={line width=.2pt,draw=gray!50},
        ymode=log,
        log origin y=infty,
        xshift=-10em,
        ytick={5, 7.5, 10},
        yticklabels={$5\times$, $7.5\times$, $10\times$},
        ylabel={Speedup (log)}]      
        \addplot[
            every node near coord/.style={
                anchor = west
                /pgf/number format/fixed,
                /pgf/number format/precision=1
                },
            nodes near coords={%
            \footnotesize {\pgfmathprintnumber[fixed]{\pgfkeysvalueof{/data point/y}}}$\times$},
            mark=*,
            color=red] coordinates {
                (1, 12.1) 
                (2, 8) 
                (3, 5) 
            };
        \addplot[
            every node near coord/.style={
                anchor = west
                /pgf/number format/fixed,
                /pgf/number format/precision=1
                },
            nodes near coords={%
            \footnotesize {\pgfmathprintnumber[fixed]{\pgfkeysvalueof{/data point/y}}}$\times$},
            mark=square*,
            color=violet] coordinates {
                (1, 11.5) 
                (2, 9) 
                (3, 6) 
            };
        
        \addplot[ 
            mark=x, 
            color=black,
            mark size = 0.3 em,
            mark options={line width=1pt},
            xshift=0.8em
        ] coordinates {(4,5)};
        \addplot[ 
            mark=x, 
            color=black,
            mark size = 0.3 em,
            mark options={line width=1pt},
            xshift=0.3em
        ] coordinates {(4,5)};
        \addlegendentry{\footnotesize SR Speedup}
        \addlegendentry{\footnotesize IL Speedup}
    \end{axis}
\end{tikzpicture}
        \caption{Detailed execution time and speed up plots replacing tables.}
    \end{figure}
\end{frame}

\begin{frame}{Manuscript A: Submission cycle 2}
    Submitted to ICS'25 (CORE Rank A). \\
    \begin{center}
    Reviews: \emph{{\color{YellowOrange} weak reject}, {\color{YellowOrange} weak reject}, {\color{red} reject}. \\ All low familiarity. }
    \end{center}
    Final decision: \emph{pending.} \\[1em]

    Key feedback: \emph{pending.}  \\[1em]

    \textbf{Next steps: Resubmit to ICPP'25 (CORE A).}
\end{frame}

\begin{frame}{Current research: Manuscript B}
    \begin{center}
        \emph{``Load balancing PDEs on APU architectures.''}
    \end{center}
    Key contributions:
    \begin{itemize}
        \item Demonstrate an application tailored to unified memory models --- not just ported over from GPU. 
        \item A performant solver for AMD MI300A. 
        \item Performance comparison between psuedo-unified memory, \emph{e.g.}, Nvidia Grace-Hopper, with execution time and power. 
        \item And between traditional discrete 
    \end{itemize} \\[1em]
    \textbf{Possible motivations} --- Head to head with SoTA solvers, \emph{i.e.,} matrix-free GPU.
    
    {\footnotesize \textsc{note:} current art is just porting not demonstrating applications tailored MI300A and is just porting them over.} \\[1em]
    
    Target: SC'25.
\end{frame}

\begin{frame}{Area Exam}
    \begin{center}
        \emph{``Current directions in high-performance PDE solvers: contemporary theory, memory efficiency, \& applications.''}
    \end{center}
    Organization 
    \begin{enumerate}
        \item Introduction \& background.
        \item PDE primer,
        \item Contemporary applied math \& solver methods,
        \item Applications \& use cases,
        \item Performance \& hardware, 
        \item Software library support,
        \item Summary.
    \end{enumerate} \\[1em]
    Date: \emph{pending}.
\end{frame}

    \begin{frame}{Future directions}
        Manuscript C (late 2025, performance track)
        \begin{center}
            \emph{`Optimizing 2-D variable coeff. PDE performance on GPUs."}
        \end{center} \\[1em]
        Manuscript D (early 2026, application track)
        \begin{center}
            \emph{``2-D variable coeff. earthquake cycle model"} \\
            --or-- \\
            \emph{``3-D earthquake cycle model."} \\ 
        \end{center}
        Manuscript E (2026, journal paper)
        \begin{center}
            \emph{``Modeling earthquake cycles on massive scales."}
        \end{center}
    \end{frame}
    \begin{frame}{Timeline}
        \begin{center}
            \captionsetup{singlelinecheck=false, font=blue, labelfont=sc, labelsep=quad}
            \caption{Timeline}\vskip -1.5ex
            \begin{tabular}{@{\,}r <{\hskip 2pt} !{\foo} >{\raggedright\arraybackslash}p{5cm}}
            \toprule
            \addlinespace[1.5ex]
            Spring 2025 & Resubmit Manuscript A (ICPP)\\
              & Submit Manuscript B (SC) \\
              & Defend Area Exam \\
            Fall 2025 & Submit Manuscript C (PPoPP/IPDPS)\\
            Winter 2026 & Submit Manuscript D (ICS/ICPP/EuroPar)\\
            Spring 2026 & Submit Manuscript E \\
            Summer/Fall 2026 & Dissertation \\
            \end{tabular}
        \end{center}
    \end{frame}

\end{document}
